For the 20th anniversary, John Carmack revisited and played the game with commentaries. Here is the transcript.\\
\par
\hrule \par
\vspace{10pt}
May 9th, 2012\\ \par

I actually had a great time just a couple years ago dragging Wolfenstein [3D] out for the iOS version where I hadn't really looked at it for over a decade probably prior to that. The core essence of what's in a first person shooter of navigating around the environment, picking stuff up, and shooting at your enemies really was there from the very beginning. Obviously it's amazing how far we've come-if you look at the old 320x200 graphics in 256-color VGA and compare it with today's games-but the core essence of taking an experience that used to just be done in top down. We were all familiar at the time with top-down shooters like Gauntlet where you would run around and shoot at the enemies. Taking very similar gameplay and projecting it into the 3D environment with perspective. The fact that that changed the way you perceived and experienced the game so much was the real triumph of what we did then.\\ \par

The evolution of our first person experience started with a game called Hovertank [3D], which was like Catacomb 3-D and Wolfenstein a three degree of freedom game where you could move around in X,Y, and rotate, but it had flat shaded walls, they were like red and green blocks, but it did have the scaled bitmaps for the enemies. Then we went to Catacomb 3-D which had the texture mapped walls as well as the scaled enemies, but it was really the absolute minimal case where it was nothing but a world full of blocks. There were blocks that you could make disappear to simulate doors, the enemies always faced you and it was pretty much just the mob of enemies that would target down on your position. It was still pretty neat, but with Wolfenstein we wanted to add several more key things to it. We added the doors as an element which moved a little bit away from the blocks that were pushed in halfway through. We added enemies that could be seen from different angles. They weren't really 3D but they had 8 different rotations, some of which were mirror flips, which also allowed us to have enemies that would walk around through the world. That's something that we sort of lost a bit when we moved to Doom where in Wolfenstein we've got all these guys moving around on paths and you get more of an opportunity to sneak up behind people. You know that you dodged an encounter by seeing them walk right past you without quite noticing you. That had several pretty neat aspects to it. So we knew that we wanted to do some of these types of gameplay enhancements or weapons, more things to pick up, but after we'd done Catacomb [3-D] the first thing that we were going to do was the game called "It's Green and Pissed". It was just going to be an Aliens game, with most of the enhancements that we wanted to do for Wolfenstein. It was only later that we latched on to the idea of taking this favored old Apple II game theme, the "Return to Castle Wolfenstein" theme. It was a wonderful thing a year or two later when we got to see Silas Warner, the original author of the Apple II Castle Wolfenstein game at one of the Apple II KansasFests that Romero, Tom, and I went to and have him give us his approval for the game that we did.\\ \par

Technology wise, the two previous 3D games, Hovertank and Catacomb 3-D, I had done those in a object space rendering where it drew limited polygons. They're one dimensional, just line segments that were restricted axially. But it had something that resembled a polygon rasterizer and a polygon clipper. Those were both done in 4-6 weeks apiece. I had really quite a bit of difficulty with it. Going back in time 20 years there weren't all the references and existence proofs and tutorials on all this. I was having a hard time getting some of that stuff to be as robust and reliable as it needed to be. You could get a few freak out cases in Catacomb 3-D and one of my real goals was to simplify it enough that it would be really rock solid and robust. The fact that it came out to also be a speed up was more a comment on the poor quality of my previous implementation, because when Wolfenstein needed to gain some speed on much poorer platforms like the Super Nintendo I went back to more of a rasterization approach with BSP trees rather than raycasting. But Wolfenstein did wind up being both more efficient and more robust than my previous implementations. This was all Wild West for me back then. I was figuring it all out as I went along. And there weren't a lot of things to look at as examples.\\ \par

I seem to remember that we had the first one running in about 3 months. We followed it up with Spear of Destiny before moving on to Doom, but it was a very short amount of time. We leveraged the toolsets that we were using to create the 2D games, the Commander Keen series, where we had a good tile editor that John Romero had developed, and since the Wolfenstein maps were basically simple tile maps we were able to make things happen really quickly on that. Also the maps were just so quick to create. We had several maps in shipping products that were literally done in one day. Somebody would go scrub out a map, we'd play it a bunch of times, tweak things a bit, and it would go into the project. It's always interesting looking back at the games where you look at the source code and it all fits in one directory. It's a handful of C files and a few .asm files; there's just not that much to it. It seems today we can't throw a dialog up on the screen without invoking 10 different frameworks and 50 thousand lines of code.\\ \par

The thing that stands out a lot was the first-and it influenced a lot of what id software did afterwards-the first sense of people tweaking with the projects, you had people figuring out how to unpack the levels. That wasn't straightforward, because we were trying to fit on floppy disks at this time so I had to develop all of this compression technology. The funny thing is, in hindsight I had sort of independently reinvented LZSS and Huffman coding in very adhoc ways. People figured all these out, extracted everything, and started making character editors and level editors. Neither of those were designed to be done, and they weren't straightforward. The characters were in this column packed format that made it more efficient to draw without transparency tests, but made it really difficult to figure out what these original pictures looked like. People started doing these really neat things with it. Once we recognized that there were a large body of people that wanted to do this that influenced a lot of our future decisions in Doom and Quake, about making it actually easy and straightforward and encouraging people to undertake those things.\\ \par

I have clear recollections to this day and especially at that time, thinking back to when I was getting into gaming, when I was teenager, about how I wished I could have that kind of access to the inside and guts of the games that I played. I can remember breaking out the Apple II sector editors to give myself lots of gold in Ultima III. Wishing that I had the ability to look at the source code for those old titles. Being able to make that type of thing come true as we were later able to release the full source code as well as the modding tools for the games has been something I'm really proud of in my career.\\ \par

One of the things that I used to insist on in the old games was that you should be able to launch the game and just bang Enter a few times to get in and be playing the game. That's something that I think is a little bit of a shame nowadays that we have to go through so many logo crawls. Of course we're trying to preload things and do effective stuff in that time, but I was always about getting into the games as quickly as possible. Being able to go in and see what the game's about and have fun really quickly. In the early days when we were more looking at the Castle Wolfenstein Apple II game, which was much more of a stealth game, we had the ability like you did in the original to drag bodies around so the guards wouldn't see them. So I can still look at that body and clearly remember the time when you could go over, hit a key, and drag it with you to some out of the way place. But as we got on we found out that the game was much more fun to be guns blazing than to be sneaking around. That's really a lesson that gets relearned continuously as the industry has involved. One of the things that was always a little bit annoying in the original game was how slidy everything was around the doors. In some other versions later on it would actually pocket in there so you could find your way a little bit easier.\\ \par

I remember the push that we wanted to get something else and couldn't we get something like the doors, and that certainly was on the ugly side on the code. The way it was hacked in with the ray tracers, but it was certainly a worthwhile thing. But you had so many people that would play the game with their nose scrubbing up against the wall like this. It is interesting, I have lots of memories of the later versions of the games where especially in this room when I did the Atari Jaguar version I remember finding that in this room looking at about this point I could actually go 60 frames per second, which was unseen before in 3D gaming. It was so cool but of course I couldn't hold it for the rest of the game and I had to lock it down at 30 frames per second. That's still one thing that I clearly remember about that part of that room, just feeling that silky smooth 60hz gameplay while on the PC we were still struggling along, like this is in the teens for the most part.\\ \par

I personally did the PC version, the Super Nintendo version, and the Jaguar version. Wolfenstein, kind of like Doom, anything that's got a CPU has a version of Wolfenstein on it since we had the open source versions, but we had official versions on the Mac which had higher quality graphics. A few of the other platforms, the [Acorn] Archimedes RISC machine, several other things that we had licensing agreements with. On the Super Nintendo version, you can't shoot dogs in a Nintendo game, at least back then, so the dogs had to be changed to evil German rats, to go along with the fact that the blood had to be changed to green, because you can't actually shoot people, so they had to be some kind of aliens wearing human uniforms or something.\\ \par 

This was our first game with digitized sound effects. The music is still frequency modulated AdLib Synth, but we had digital sound effects. Working with Bobby Prince for the music [who created] all the sound effects for us. If you had the AdLib you just got these squirrely not so interesting little beeps, but if you had the Sound Blaster you got the digitized sound effects.\\ \par 

There's two graphical glitches, one that I just saw there, there was a missing column as I hit the push wall, and occasionally as I'm moving around I may notice one stretched column that shows up. They were constrained in that they wouldn't cause crashes or difficulties, although here you can see one of the graphical artifacts. Normally when you're a decent distance from things, everything smoothly expands, but when you get up close and they start going off the edge of the screen, you start getting more quantization. That's a result of the fact that the core graphics technology behind the texture mapping in this timeframe was compiled scalars where there's actually a different section of assembly language code that was programmatically generated to draw a 64 tall piece of graphics, two pixels tall, four pixels tall, six pixels tall, and so on all the way up to the full height of the screen, but it started taking up too much memory and it would run out in a 640K system if I let them stretch all the way up to the largest possible scale that you'd get there in steps by one so it started taking some shortcuts and saving a little bit as it got bigger.\\ \par

Another interesting thing in the timeframe of Wolfenstein was we were still very strongly influenced by arcade games and the arcade tradition. All our previous games, with Commander Keen, and the games that we'd done at Softdisc, they still had the concept of lives as if you're putting quarters in you get three tries and then you have to put another quarter in. It was only with Doom where we made the realization, obvious in retrospect, that you're not in an arcade, you just want to play the game to have fun, and we finally gave away the concept of lives. In Wolfenstein you still had 1-Ups hidden some places, and you had to, in theory, get through the game without losing all of your lives. There were still the rewarding times in the game with the secret rooms with the acres of bonus items that you get to run over.\\ \par 

I still regret the fact that we were a little late to the game on iOS. When it first came out I pitched that we should move Orcs \& Elves, one of our successful feature phone titles. We got something up and running but we couldn't pull the trigger on actually turning it all the way into a product. It was a over a year after the iPhone came out when I finally got down and said that I wanted to go and actually try out the first person shooter stuff on there. I think it was less than a month that it wound up taking me. A lot of that was learning the ins and outs of iOS. Cass Everett had helped me get a framework up and going there. It was interesting to figure out what base I wanted to work with. The original Wolfenstein 3D was 16-bit code, it had a lot of x86 assembly language, it didn't have C fallback paths. This was before we had discovered cross platform development and all the benefits of doing things that way. It would have been a chore to go rewrite it, although it's not that much code. I could have gone in and literally gone over every line of the code in not too terribly long, but this was open source and the internet to the rescue where after I'd released the original Wolfenstein source code a number of people had gone and done additional work on top of it. Somebody had made a nice 32-bit clean version of the game that was portable, used OpenGL for graphics. It made the task of getting it onto iOS a lot easier. Interestingly it introduced a couple new bugs that weren't there in the original. When I was debugging things I would see some enemy behavior that turned out to be a bug in the conversion from 16-bit to 32-bit. That was a fun project. After that doing Doom Classic was also a lot of fun.\\ \par

I remember how challenging it all way, way back then. Of course in retrospect, with 20 years of experience going through different things you know how you should do things, or how you should have done it back in the past. Probably even more so with Doom where having done a number of versions of Doom over the years I knew exactly how I really should have structured the loop, especially for a hardware accelerated game. One of the funny things, or interesting at least, with the case of Wolfenstein and really all our games to some degree, is that when we're actually making the games I usually haven't thoroughly played through every level of the game during development. I'm busy going through the same levels over and over again, working out problems, optimizing things. I can remember during the Jaguar Wolfenstein development  when I was all happy with it running it at a nice 30 frames per second, good performance, I was saying, "wow, this is really lots of fun". Romero gave me some kind of snarky comment, like "it was lots of fun in the PC version too, you just didn't play that one that much". I see that even on some of the later games, when we go back and look at previous games I'll notice things that have been there forever but I just never picked up on them the first time through.\\ \par

I'm sure a modern gamer that didn't come up through the generations of all of it might be a little hard pressed to see what the fuss was all about, but especially if you took one of the versions that was running at a high frame rate, and had good responsive controls, you could still get the same basic vibe. I think the Jaguar version was probably my favorite because it had shoulder buttons to be able to move around so you get the strafing action more easily, it ran at a consistent framerate, things were a little bit crisper in some ways. I had a lot of fun on that and I think that somebody today would have a similar sense to going back and looking at a game like Defender or Robotron [:2084], one of the classic arcade games, that you can still see the fun in today. It's also clear, obviously, it's hard to find examples of anything else where you have progressed as much as we have. When you look at the delta from Wolfenstein to Rage and you put them side by side. It's true that I couldn't have imagined 20 years ago what we would be able to do in the games today, what Rage could look like, because computer generated graphics in the movies 20 years ago didn't look as good as what we do real time in the games now. That sense of 6 orders of magnitude of progress, a factor of a million to do really sort of the same basic game. You run around, you pick stuff up, you engage your enemies and defeat them and move on to the next level. But we're throwing a million times more power at it now than what we had here. It's a pretty good use for it, the games are a whole lot better, they're more interesting in all sorts of ways.\\ \par

Probably the gameplay thing that I most remember in Wolfenstein as a positive sense is when you wind up going through a door and there's an S.S. right on the other side with a machine gun, and you're pumping the machine gun shots into them and they're doing the side to side twitch back and forth. That's my quintessential Wolfenstein moment. My quintessential Doom moment is the Imp getting a faceful of blood bursts from a shotgun. Most games have that thing that I associate with the action. Wolfenstein had a lot of good play value in it. There were a lot of levels because they were easy to create, and quite a bit of playtime through it. Of course when people started making their own levels you had as much of it as you could possibly want to play.\\ \par

It's been a long road on the technical side. Certainly the results, but also what we wind up doing to go through and make it. It's interesting to go back and see how much we were able to accomplish with so little. In many cases there's so much waste that's endemic in development, both in how you spend your time, how you spend your resources, for what you get out of it. When we look back at what we were able to do in a small number of months with a single directory of code files, how much enjoyment we were able to give people out of it, it's a little bit of a cautionary tale about how everything seems to take a schedule of meetings and a plan and a framework of objects. Sometimes you wind up just spending a whole lot of work for things that could probably have been done in a simpler fashion and wind up achieving similar goals.\\ \par 

I think that any big company doing AAA development on a multiyear cycle, there is something that they've lost by not having that mad dash to get something done in a small number of months, to be able to find the real value and get it into a space where it can be tested and improved upon. It's a conscious trade, the magnificent majestic things that we do today just aren't possible to hack together, but the fun core elements of things aren't the ones that take a man century of effort like so many of the big things do. I think it's good to be reminded every now and then that the essence isn't necessarily in what you throw the man centuries and huge human waves at.
