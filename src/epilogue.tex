After the release of Wolfenstein 3D, id Software kept itself busy and became a video game industry powerhouse. id went on to publish ten more games over the next two decades. 
 \begin{figure}[H]
\centering  
\begin{tabularx}{\textwidth}{ X  Y Y }
  \toprule
  \textbf{Name} & \textbf{Engine}  & \textbf{Release date} \\
  \toprule \codeword{Doom} & idTech 1 & December 1993 \\ 
           \codeword{Doom II} & idTech 1 & September 1994 \\ 
           \codeword{Quake} & idTech 2 &June 1996 \\ 
           \codeword{Quake II} & idTech 2 & December 1997 \\ 
           \codeword{Quake III} &  idTech 3 & December 1999 \\ 
           \codeword{Doom III} &  idTech 4 & August 2004 \\ 
           \codeword{Wolfenstein 3D: iOS} &  idTech X & March 2009 \\ 
           \codeword{Rage} &  idTech 5 & October 2011 \\ 
           \codeword{Doom 3: BFG Edition} &  idTech 4.5\protect\footnotemark & October 2012 \\ 
           \codeword{Doom} &  idTech 5 & May 2016 \\ 
  \toprule
\end{tabularx}
\caption{id Software titles.}\label{fig:vga_history}
\end{figure}
\par
\footnotetext{idTech 4 boosted with subsystems from idTech 5 such as the job system to take advantage of multi-core architecture.}
Not only did the team drive technological progress by producing new game engines, they also heavily influenced the industry by licensing their technology throughout the early 90s. A rough estimate accounts for 80 games powered by id Software's engines.\\
\par
Beyond software, id heavily influenced the graphic hardware industry. With VQuake and GLQuake offering support for 3D acceleration cards, id helped to drive sales for Rendition's Verite 1000, 3dFx's Voodoo, and later PowerVR. Doom 3 famously used OpenGL at a time where many studio succumbed to Microsoft's DirectX API.\\
\par
id remained true to its Right Thing to Do roots and continued releasing the source code of each previous engine, helping countless programmers educate themselves with recent technology. Sadly, this practice stopped with the Rage (id Tech 5) engine.\\
\par
 \begin{figure}[H]
\centering  
\begin{tabularx}{\textwidth}{ X  Y  Y}
  \toprule
  \textbf{Source Code name} &  \textbf{Release date} & \textbf{Delta from game}\\ 
  \toprule \codeword{Wolfenstein 3D} & July 21, 1995 & 3 years, 2 months\\ 
           \codeword{Doom} & December 23, 1997 & 3 years, 2 months\\ 
           \codeword{Quake} & December 21, 1999 & 3 years, 6 months\\ 
           \codeword{Quake II} & December 21, 2001 & 4 years, 0 months \\ 
           \codeword{Quake III} & August 19, 2005, & 5 years, 8 months\\ 
           \codeword{Doom III} & November 22, 2011 & 7 years, 3 months\\ 
           \codeword{Wolfenstein 3D: iOS} &  March 25, 2009 & Same day \\ 
           \codeword{Doom 3: BFG Edition} & October 2012 & Same day \\  
  \toprule
\end{tabularx}
\caption{id Software source code releases.}\label{fig:vga_history}
\end{figure}
The Wolfenstein 3D franchise lives on to this day, having been licensed to numerous studios. At E3 2017, Bethesda announced Wolfenstein II: The New Colossus, a sequel to The New Order. At the time of the writing of this book, it is scheduled for release on October 27, 2017.\\


\section{Where Are They Now?}
\par
\textbf{Tom Hall} left id Software shortly after the release of Spear of Destiny, during the development of Doom in 1992. He worked at Apogee and later Ion Storm and was involved in some great games of the 90s such as Rise of the Triad, Terminal Velociy and Anachronox. Tom now lives in the San Francisco Bay Area and is Senior Creative Director of mobile games at Glu's GluPlay studio, producing titles such as Cooking DASH, Diner DASH, and Gordon Ramsay DASH.\\
\par

\textbf{John Romero} left id Software after the release of Quake. He contributed massively to Doom/Quake and also developed the licensing business branch of id. His "all in-hand"\footnote{Engine + Tools recommendation (NeXT) + Mentoring.} solution based on the id Tech 1 engine (the same powering Doom) notably helped Raven publish Heretic and Hexen. He later founded other companies such as Ion Storm and Monkeystone Games, the latter of which was a precursor to the field of mobile game development. He now lives in Ireland and is working on an upcoming title called Blackroom.\\
\par

\textbf{Adrian Carmack} left id Software after the release of Doom III and retired from the video game industry. He lives in USA and recently announced his collaboration with John Romero on Blackroom.\\
\par

\textbf{John Carmack} remained with id Software until 2013, after which he joined Oculus VR as CTO. He has received several awards for his accomplishments, including two Emmmy awards for "Science, Engineering \& Technology ", a Game Developers Conference Lifetime Achievement, and a BAFTA Fellowship Award.\\
\par
\textbf{Jay Wilbur} worked on Doom, Doom II, Final Doom, and Quake. He left to become Vice President of Business Development at Epic Games.\\
\par
\textbf{Kevin Cloud} is the last man standing at id Software of the original Wolfenstein 3D team. He was an owner until Zenimax Media's acquisition of the studio and is now an Executive Producer.
\par
