\documentclass[book.tex]{subfiles}
\begin{document}


For the past 10 years, I have been writing articles explaining the internals of game engines. It all started back in 1999 when I downloaded the freshly open sourced code of \mbox{Quake} and eagerly opened it with \codeword{Visual Studio .NET}. After a few days I deleted \codeword{quakesrc} folder, unable to make sense of anything and discouraged.

\bigskip

A few years later I came across legendary programmer and technical writer Michael Abrash's Graphics Programming Black Book. His marvellous articles gave away the big picture of Quake engine and detailed some of its algorithms and data structures. Equipped with this precious knowledge I went back to the code and understood it to the deep down.

\bigskip

I thought many other programmers may be like me: Capable but discouraged by apparent complexity. So I started to write ``source  code reviews'' and published them on 
\href{http://fabiensanglard.net}{\textit{fabiensanglard.net}}. Over the years, I proceeded to publish more than fifty articles, selecting legendary games such as Doom, Quake or Out Of This World. I would open the engine, explore the subsystems, understand the overall architecture and draw a map that hopefully sparkled interest and saved time to other adventurous programmers.

\bigskip

Sharing my knowledge was a rewarding experience. Not only the feedback from readers was overwhelmingly positive, to try to explain something in simple terms was also a way to make sure I mastered a topic. It also allowed me to become a better teacher and learn to rely extensively on good drawings. A picture
is worth more than a thousand words\footnote{"Code: The Hidden Language of Computer Hardware and Software" by Charles Petzold is a superb example.}. 

\bigskip

Eventually I decided to take my articles to the next level and came up with this book which I hope will be the first of a series about revolutionary  engines. I try to pick software pushing the hardware to its boundaries and explain how they interact.

\bigskip

It may seem like a waste of time to read those ``old'' source codes dedicated to obsolete hardware, compilers and operating systems. But there is tremendous values in them. Not only they teach clean design, provide a deeper understanding of modern hardware and software, trigger new ideas and inspire innovation, they remind us of the constraints programmers from the past had to overcome. Since most engines I studied come from the same company it is also interesting to see the evolution of code style and overall mastery.

\bigskip

 But above all, the code and the assets are a testimony to the talent and determination it once took to make a dream become true. Something we all aspire to at some point in our life. 
\bigskip

To read this code made reminded me that others before me followed the difficult path of excellence. And when confronted to obstacles they managed to pull through. It inspired me to emulated them and emulate myself. I hope it will benefit you as well.
\end{document}