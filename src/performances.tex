\section{Performance}
With all the tricks and optimizations described, how well did the game run on most people's machines? DOSBox allows for emulating an old CPU and gives the following figures.

\begin{figure}[H]
\centering
\begin{tabularx}{\textwidth}{ X X X }
  \toprule
  \textbf{CPU} & \textbf{Frequency (Mhz)} & \textbf{Frames per seconds} \\ \bottomrule
  286 & 10 & 7 \\
286 & 16 & 12 \\
386SX & 25 &  16 \\
386DX & 25 & 24 \\
386DX & 33 & 30 \\ \bottomrule
\end{tabularx}
\end{figure}

To compensate for differences of power, the engine can reduce the 3D canvas to reduce the number of rays cast and the number of pixels to render on screen.
\par
Max: 304 rays resulting in 46208 pixels to render (304*152).\\
Min:  64 rays resulting in 2432 pixels to render (64*38).\\

  \begin{figure}[H]
\centering
 \fullimage{adjust_view/100.png}
 \caption{Max view: 304 rays, resolution of 304x152}
 \end{figure}
 \par

   \begin{figure}[H]
\centering
 \fullimage{adjust_view/50.png}
 \caption{Max view: 224 rays, resolution of 224x112}
 \end{figure}
 \par

   \begin{figure}[H]
\centering
 \fullimage{adjust_view/small.png}
 \caption{Min view: 64 rays, resolution of 64x38}
 \end{figure}
 \par

In 1994, Rise Of the Trial (a.k.a ROTT), also directed by Tom Hall was published by 3D Realms. It used the Wolfenstein 3D engine and also featured this window adjustment system. True to the line of the humor found in early 90s video games, it features an easter egg teasing the player to buy a 486 when the lowest setting is chosen.
    \begin{figure}[H]
\centering
 \fullimage{adjust_view/rott.png}
 \end{figure}
 \par