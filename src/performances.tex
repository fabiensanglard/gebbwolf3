\section{Performances}
With all the tricks and optimization described, how well did the game run on most people desk? DOSBox allows to emulate old CPU and gives the following figures.

\begin{figure}[H]
\centering
\begin{tabularx}{\textwidth}{ X X X }
  \toprule
  \textbf{CPU} & \textbf{Frequency (Mhz)} & \textbf{Frames per seconds} \\ \bottomrule
  286 & 10 & 7 \\
286 & 16 & 12 \\
386SX & 25 &  18 \\
386DX & 25 & 24 \\
386DX & 33 & 30 \\ \bottomrule
\end{tabularx}
\caption{Component quality for a game engine.}
\end{figure}

To compensate for the differences of raw power, the engine allows to reduce the 3D view which reduces the number of rays to cast and the number of pixels to render on screen.
\par
Maximum: 304 rays and 46208 pixels (304*152).\\
Minimum:  64 rays and  2432 pixels (64*38).\\

  \begin{figure}[H]
\centering
 \fullimage{adjust_view/100.png}
 \end{figure}
 \par

   \begin{figure}[H]
\centering
 \fullimage{adjust_view/50.png}
 \end{figure}
 \par

   \begin{figure}[H]
\centering
 \fullimage{adjust_view/small.png}
 \end{figure}
 \par

Raise Of the Trial (a.k.a ROTT) was directed by Tom Hall for 3D Reams. It used Wolfenstein 3D engine and also features the window adjustment system. In the line of the humor of the early PC games, it features an easter egg when the player chose the lowest setting. It teased the player to buy a PC with a 486 CPU!
    \begin{figure}[H]
\centering
 \fullimage{adjust_view/rott.png}
 \end{figure}
 \par