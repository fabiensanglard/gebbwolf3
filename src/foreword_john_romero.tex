There are few books like the one you're holding. This is a rare technological, archeological dissection of an old, but significant, game -- Wolfenstein 3D, the grandfather of the first-person shooter genre. There are a couple other books that dive deep into the tech of old games such as Retrogame Archeology, but no other book explains in exhaustive, technical detail the computing environment that the game was written in before attempting to explain how the subject worked. No other book details the source code of a game with such a thorough and complete understanding that there is no room for questions as to how it operates.\\
\par
Fabien has been complete and thorough in his analysis of this important artifact. He explains the struggles we faced in those days to raycast and render a scene at 70 frames per second. Along the way he delivers plenty of development nuggets of information that have been revealed only by direct questioning of the only two coders on the game, John Carmack and me. The book Masters of Doom gives a cursory glance at the making of Wolfenstein 3D, but if you want the real story, this is where you get it.\\
\par
We did not know that we were making a game that would kickstart an industry back in 1992. From the start of the project to launching the shareware version it only took us four developers four months of effort. Back then we worked at a feverish pace. Before Wolfenstein 3D, we spent a couple years making games and shipping them within two months. We had a lot of practice. When the plan to make our own, first FPS came up, my idea to retell the Castle Wolfenstein story felt like an obvious good idea. The majority of us four loved the original Silas Warner classic on the Apple II. Recreating the game in 3D felt like it could really work, so we made the decision over the course of probably an hour, left our Commander Keen 7 dev directory (never to return), and created the W3D directory in January 1992.\\
\par
Making Wolfenstein 3D, first in Madison, Wisconsin, then in sunny Mesquite, Texas a couple months later, was a lot of fun. We had so many adventures in such a compressed space of time back then. We even flew out to meet Ken and Roberta Williams at Sierra, and showed Ken our fledgling homage to the great original that Ken revered as well. Wolfenstein 3D got Ken Williams to offer \$2.5 million to buy our small company. The deal didn't work out, luckily, and we continued to make the game. We moved the company from the north to the south. We made our first VGA game.\\
\par
After launch, we spent another two months putting together episodes 2-6. It was a great, happy time. So much happened during the development of the game, and after its launch everything in our world felt like it does when you finally reach the stars, relatively-speaking. DOOM would take us to another galaxy soon enough.\\
\par
Enjoy this excellent book. It holds the complex secrets to our first real shooter. It may not be easy to understand, but we needed to know all this information in 1992, that Fabien is teaching you now, so we could create the FPS.\\
\par
Cheers,\\
\par
John