Wolfenstein 3D was released in May 1992 by id Software. The game  became an overnight phenomenon. Reviewers were ecstatic, gamers loved how fast and fun the game was and thanks to the shareware model it was widely distributed.

\bigskip

By the end of 1993 the game had sold more than 100,000 units, bringing fame and a little bit of fortune to the team from Madison, Wisconsin.

\bigskip

Three years later, on July 21st, the company started something that would became a tradition. They released the full source code of their previous game. A rare gift that few others have dared replicating ever since.

\bigskip

Programmers all over the world jumped on it, eager to discover the secrets of the game engine. Not only did they take a look under the hood, they also ported it to other systems and maintained it, protecting it from software erosion. Twenty years later Wolfenstein 3D can still be played on pretty much anything equipped with a microprocessor !

\bigskip

While reading the source code, I grew fascinated by the way the software tamed the hardware:  Machines of the early 90s were running with DOS 4.0 operating system on Intel processor i386SX / i386DX clocked from 16Mhz to 33Mhz and equipped with 2MB of RAM. Those machines were difficult to program with their convoluted memory model, lack of floating point unit and the universal yet limited VGA display. 

\bigskip

You had to be driven by passion to program games on those things. After all, even the people building the hardware would have said that what id Software was trying to achieve was not what the machines were meant to do. This is where the true beauty of Wolfenstein 3D is: For every obstacles, the id Software team found a way to not only circumvent the limitations but often turn them into advantages.

\bigskip

 \textbf{\underline{Disclaimer :}} The description that follows is technical and will appeal mostly to programmers. If you are more into the human aspect of game programming I would recommend instead to read David Kushner’s chef d’oeuvre: “Masters of Doom: How Two Guys Created an Empire and Transformed Pop Culture”.
This book tries to give a detailed story of how Wolfenstein 3D was done. It is divided in three parts reviewing the hardware, the team and the software.